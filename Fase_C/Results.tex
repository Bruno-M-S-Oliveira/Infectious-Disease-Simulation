\section{Resultados e Conclusão}
Com o programa desenvolvido, é possível o seu utilizador ajustar vários parâmetros disponíveis e realizar comparações, nomeadamente entre diferentes países e diferentes regiões.
Da mesma forma, é possível modificar tanto $R_0$ como os parâmetros da doença em estudo (latência, período infeccioso e período de imunidade).
A modificação dos vários parâmetros da vacinação permite testar várias estratégias possíveis de implementar e, concluir a que melhor se aplica à realidade de cada situação.
Tendo em conta a população inserida em cada grupo etário, e as suas suscetibilidades e probabilidade de morte intrínsecas, foi possível delinear simulações no que toca à evolução da pandemia COVID-19 em Portugal.

%\begin{multicols}{2}
%        \begin{figure}[H]\centering
%                \includegraphics[width=\linewidth]{Images/Result.png}
%                \caption{Resultados}
%        \end{figure}
%\end{multicols}
