\section{Modelo SEIRDS}
O modelo SEIRDS divide a população em estudo com base no contacto prévio com o vírus nos seguintes grupos:
\begin{itemize}\begin{multicols}{3}
        \item Susceptível ($S$)
        \item Recuperado ($R$)
        \item Exposto ($E$)
        \item Morto ($D$)
        \item Infetado ($I$)
\end{multicols}\end{itemize}

Na \ref{fig:Model}, encontra-se esquematizada a progressão que pode ocorrer entre os vários grupos ao longo do tempo e que de seguida será descrita matematicamente.
\begin{figure}[H]\centering
        \includegraphics[width=.55\linewidth]{Images/SEIRDS.pdf}
        \caption{Esquema do modelo SEIRDS sem vacinação.}
\label{fig:Model}\end{figure}

\subsection{Descrição Matemática}
Para descrever matematicamente o modelo proposto, irá considerar-se a progressão de um indivíduo inicialmente susceptível ao longo do tempo.
Em média, este irá contactar diariamente com $C$ pessoas, sendo que uma fração $\frac{I}{N}$ estará infetada com $N\equiv S+E+I+R$ a população total.
Definindo-se susceptibilidade $u\in[0,1]$ como a probabilidade do contacto com uma pessoa seropositiva levar ao desenvolvimento da doença, a probabilidade do indivíduo que se está a seguir passar de susceptível a exposto num determinado dia é $\Lambda$:
\begin{equation*}
        S \xrightarrow{\Lambda\vdot S} E
        \qquad\qquad
        \Lambda \equiv u\vdot C \vdot \frac{I}{N}
\end{equation*}

Durante o período de incubação do vírus, um indivíduo exposto não irá apresentar sintomas ou infetar terceiros.
Assumindo que a duração deste período é uma variável aleatória com distribuição exponencial e valor médio $\tau_E$, o número de pessoas expostas que passam a estar infetadas num dado dia é:
\begin{equation*}
        E \xrightarrow{\alpha\vdot E} I
        \qquad\qquad
        \alpha \equiv \frac{1}{\tau_E}
\end{equation*}

Uma vez infetada, uma pessoa permanecerá nesse grupo durante um determinado período de infeção, durante o qual poderá contagiar terceiros.
Ao fim do período de infeção, o indivíduo afetado poderá morrer ou recuperar, passando então para os respetivos grupos.
\begin{equation*}
        \begin{gathered}
            I \xrightarrow{\gamma\vdot I} R \\
            I \xrightarrow{\mu\vdot I} D \\
        \end{gathered}
        \qquad\qquad
        \gamma \equiv \frac{1-IFR}{\tau_I}
        \qquad
        \mu \equiv \frac{IFR}{\tau_I}
\end{equation*}
em que $IFR$ é a mortalidade da doença e se assumiu que a duração do período de infeção é uma variável aleatória com distribuição exponencial e valor médio $\tau_I$.

Após recuperar do vírus, adquire-se imunidade natural que previne infeções subsequentes.
Assumindo que a duração da imunidade é uma variável aleatória com distribuição exponencial e valor médio $\tau_R$.
\begin{equation*}
        R \xrightarrow{\omega\vdot R} S
        \qquad\qquad
        \omega \equiv \frac{1}{\tau_R}
\end{equation*}

\subsection{Faixas Etárias}
Um dos fatores com maior influencia sobre o desenvolvimento de uma doença contagiosa em cada pessoa é a sua idade.
Para ter isto em consideração, o modelo SEIRDS foi adaptado para uma população distribuída por 8 faixas etárias de $10$ anos e uma para acima dos $80$ anos ($[0,9]$, $[10,19]$, ..., $[70,79]$, $[80,\infty[$).

Com esta alteração, as variáveis $S$, $E$, $I$, $R$ e $D$ tornam-se vetores a 9 dimensões.
Também a susceptibilidade, $u$, e a taxa de mortalidade, $IFR$, passam a ser especificadas para cada faixa etária e o número de contactos diários, $C$, uma matriz $9\cp9$.

\subsection{Vacinação}
Para implementar a vacinação, criam-se oito novos grupos:
\begin{itemize}\begin{multicols}{2}
        \item Susceptível e vacinado ($S_v$)
        \item Exposto e vacinado ($E_v$)
        \item Infetado e vacinado ($I_v$)
        \item Recuperado e vacinado ($R_v$)
        \item Susceptível e não vacinável ($S_x$)
        \item Exposto e não vacinável ($E_x$)
        \item Infetado e não vacinável ($I_x$)
        \item Recuperado e não vacinável ($R_x$)
\end{multicols}\end{itemize}
e definem-se os seguintes parâmetros:
\begin{itemize}\begin{multicols}{2}\raggedcolumns
        \item Total de vacinas a administrar ($V_G$)
        \item Número de doses no primeiro dia ($V_0$)
        \item Aumento diário de doses ($V_d$)
        \item Eficácia da vacina ($V_e$)
        \item Faixas etárias prioritárias
\end{multicols}\end{itemize}

As vacinas são administradas no início de cada dia antes da dinâmica de grupos com as seguintes regras:
\begin{enumerate}
        \item As vacinas apenas são administradas a pessoas susceptíveis ($S$) e expostas ($E$)
        \item Em cada dia $i$ disponibilizam-se $(V_0+V_d\vdot i)$ doses até que se tenham adminstrado no total $V_G$ doses
        \item Se as faixas etárias prioritárias tiverem membros por vacinar, estas serão vacinadas equitativamente
        \item Se sobrarem doses, as faixas etárias não prioritárias serão vacinadas equitativamente
\end{enumerate}

As pessoas vacinadas irão então seguir uma dinâmica de grupos em tudo análoga à já descrita sendo que, naquelas em que a vacina for eficaz, quando infetadas não irão apresentar sintomas ou transmitir a doença.
São então necessárias as seguintes (re)definições:
\begin{gather*}
        \Lambda \equiv u\vdot C\vdot \frac{I + (1-V_e)\vdot I_v + I_x}{N}
        \qquad
        \gamma_v \equiv \frac{1-(1-V_e)\vdot IFR}{\tau_I}
        \qquad
        \mu_v \equiv \frac{(1-V_e)\vdot IFR}{\tau_I}
        \\
        N \equiv S + S_v + S_x + E + E_v + E_x + I + I_v + I_x + R + R_v + R_x
\end{gather*}

\subsection{Equações Diferenciais}
O modelo descrito encontra-se esquematizado na \ref{fig:ModelVax} e pode ser descrito pelo seguinte sistema de equações diferenciais.
\begin{equation*}
        \begin{aligned}
                \Dot{S}   &= - \Lambda S + \omega R
                \\
                \Dot{S_v} &= - \Lambda S_v + \omega R_v
                \\
                \Dot{S_x} &= - \Lambda S_x + \omega R_x
        \end{aligned}\quad
        \begin{aligned}
                \Dot{E}   &= \Lambda S - \alpha E
                \\
                \Dot{E_v} &= \Lambda S_v - \alpha E_v
                \\
                \Dot{E_x} &= \Lambda S_x - \alpha E_x
        \end{aligned}\quad
        \begin{aligned}
                \Dot{I}   &= \alpha E   - (\gamma   + \mu)   I
                \\
                \Dot{I_v} &= \alpha E_v - (\gamma_v + \mu_v) I_v
                \\
                \Dot{I_x} &= \alpha E_x - (\gamma   + \mu) I_x
        \end{aligned}\quad
        \begin{aligned}
                \Dot{R}   &= \gamma I     - \omega R
                \\
                \Dot{R_v} &= \gamma_v I_v - \omega R_v
                \\
                \Dot{R_x} &= \gamma I_x   - \omega R_x
        \end{aligned}\qquad
        \Dot{D} = \mu I + \mu_v I_v + \mu I_x
\end{equation*}

\begin{figure}[H]\centering
        \includegraphics[width=.7\linewidth]{Images/SEIRDS_Vax.pdf}
        \caption{Esquema do modelo SEIRDS com vacinação.}
\label{fig:ModelVax}\end{figure}

\subsection{Limitações}
O modelo descrito apresenta algumas limitações.
O requisito de que população em estudo seja muito grande é inerente ao próprio modelo.
A utilização de vacinas com mais do que uma dose poderia ser resolvida com a implementação de mais grupos.
Os efeitos da sobrelotação dos hospitais e das mitigação da pandemia poderiam ser estimados a partir de situações prévias e incorporadas na simulação.
